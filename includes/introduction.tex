\indent V súčasnosti zohráva GPS dôležitú úlohu pri sledovaní polohy používateľov. Či už sa jedná o GPS údaje zozbierané inteligentným mobilným zariadením, inteligentnými hodinkami, automobilom alebo iným zariadením. Zozbierané údaje sú užitočné na niekoľko rôznych prepojení, napríklad navigácia a mapovanie, aktivita a trasy alebo analytika a dátový výskum. 

% TODO este rozpisat mapovanie a tie veci vyššie

Dáta získané GPS senzormi však nie sú dostatočne presné, pretože zariadena, ktoré snímajú GPS údaje nie sú presné. GPS snímače v mobilných zariadeniach nie sú určené primárne na toto využitie \cite{993780}. Trajektórie (tiež nazývané ako trasy) zobrazené nepresnými senzormi vyzerajú neprehľadne a kvôli svojej neprehľadnosti neodovzdajú takmer žiadnu informáciu. 

V našej práci sa budeme zaoberať tým, ako nepresné trasy upraviť a zobraziť tak, aby používateľ dokázal zistiť pohybové vzory pohľadom. Trasy bude možné zobraziť na interaktívnej mape, ktorá sa bude dať priblížiť a oddialiť. Po ich nahratí do webovej aplikácie, ich bude možné zobraziť všetky naraz alebo jednotlivo.

Práca bude obsahovať oboznámenie sa s problematikou vývoja webových aplikácii, výber vhodného vyvojového prostredia a výber vhodného programovacieho jazyka, v ktorom budeme webovú aplikáciu programovať. Ďalej obsiahne nejaké už existujúce riešenia, v ktorých analyzujeme chyby a nedostatky, ktoré sa budeme snažiť v našom riešení odstrániť. Urobíme návrh riešenia a implementujeme aplikáciu. Vo finálnej forme aplikáciu otestujeme, či funguje správne a podľa očakávaní.

Cieľom práce je vytvoriť webovú aplikáciu, do ktorej používateľ nahrá GPS trajektórie, ktoré budú spracované naším algoritmom a trasy budú pripnuté k cestnej sieti. Trasy budú ležať len na cestnej sieti, čím sa zvýši celková prehľadnosť. Webová aplikácia následne zobrazí mapu s jednotlivými trasami. Zobrazené trasy budú informovať o názve nahraného súboru, názve trasy a času nahrania.
