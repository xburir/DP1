\section{Zhodnotenie}
\indent \indent V tejto sekcií predstavíme výsledok našej práce. Webovú aplikáciu sme nasadili na verejný server, aby sme ju mohli odovzdať na testovanie reálnym používateľom. Týmto spôsobom získame objektívnu odozvu a hodnotenie našej aplikácie.
\subsection{Výsledok práce}

\indent \indent Vytvorili sme webovú aplikáciu pomocou knižnice \textit{Node.js}. Naštudovali sme trasovací engine \textit{Valhalla} a spôsoby jeho použitia. Funkcie, ktoré \textit{Valhalla} poskytuje sa nám podarilo integrovať do našej webovej aplikácie \textit{Routiak}. 

\indent \textit{Routiak} je webová aplikácia, pomocou ktorej si môže používateľ vizualizovať vlastné trasy na mape prehľadne. To znamená, že aj pri veľkom počte trás (10 a viac) prechádzajúcich tým istým miestom, môže používateľ nastaviť zobrazenie tak, aby bolo prehľadné a na mape nevznikali tzv. ``špagety'' (obr.\ref{fig:spagety}). Vďaka správnemu nastaveniu zobrazenia je možné vidieť trasy a cestnú sieť prehľadne. Používateľ sa môže v aplikácii registrovať a prihlásiť a teda je možné nahrané trasy zobraziť neskôr bez nutnosti znova trasy do aplikácie nahrať. Trasy sa do aplikácie nahrávajú v ZIP súbore, čo umožňuje nahrať viac trás súčasne. Trasy je možné zobraziť na mape jednotlivo, alebo všetky naraz obsiahnuté v ZIP súbore. Aplikácia ponúka map-match funkcionalitu, čo znamená, že trasy budú pripnuté na cestnú sieť a tým sa zlepší viditeľnosť mapy a trás na mape, obrázok \ref{fig:niespagety}. 
\newline Výhody:
\begin{itemize}
  \item map-match, pripnutie trás na cestnú sieť pre zlepšenie viditeľnosti
  \item možnosť nahrať viac trás naraz
  \item zobrazenie vlastných trás po nahratí bez nutnosti nahrávať trasy pri každom zobrazení
  \item možnosť nahrať trasy vo viacerých formátoch (CSV,GEOJSON,GPX)
  \item neobmedzený počet zobrazení mapy s neobmedzeným počtom nahraných súborov (počet nahraných súborov je obmedzený veľkosťou vyhradeného miesta na serveri)
  \item do jednej požiadavky na map-match môže ísť až 20 tisíc súradníc
  \item informácia pre používateľa v prípade, že niektorá z trás nemala požadovaný formát 
\end{itemize}
Nevýhody:
\begin{itemize}
  \item trasa môže obsahovať maximálne 20 tisíc bodov
  \item nie je možné pripnúť trasu zaznamenanú rôznymi typmi dopravy (auto + pešia chôdza)
  \item priebeh map-match ZIP súboru nie je zobrazený, iba informácia po dokončení
  \item pri nahratí jednej trasy je nutné vložiť trasu do ZIP súboru so špeciálne určenou štruktúrou
\end{itemize}
\begin{figure}[H]
  \centering
  \includegraphics[width=0.7\textwidth]{img/map-match rozdiel/pred map-match.png}
  \caption{Zobrazenie väčšieho množstva trás na rovnakom mieste}
  \label{fig:spagety}
\end{figure}
\begin{figure}[H]
  \centering
  \includegraphics[width=0.7\textwidth]{img/map-match rozdiel/po map-match.png}
  \caption{Zobrazenie väčšieho množstva trás upravených map-match funkciou na rovnakom mieste}
  \label{fig:niespagety}
\end{figure}

\pagebreak
\subsection{Hodnotenie práce reálnymi používateľmi}

Aplikáciu sme poslali ôsmim testérom, ktorí aplikáciu otestovali na vlastných trajektóriách. Pripravili sme dotazník s ôsmimi otázkami, ktorý sme dali testérom vyplniť. Otázky a odpovede je možné vidieť nižšie. 
\begin{figure}[H]
  \centering
  \includegraphics[width=.7\textwidth]{img/dotaznik/1.png}
\end{figure}
\begin{figure}[H]
  \centering
  \includegraphics[width=.7\textwidth]{img/dotaznik/2.png}
\end{figure}
\begin{figure}[H]
  \centering
  \includegraphics[width=.7\textwidth]{img/dotaznik/3.png}
\end{figure}
\begin{figure}[H]
  \centering
  \includegraphics[width=.7\textwidth]{img/dotaznik/4.png}
\end{figure}
\indent Ktorá operácia nebola vysvetlená v príručke?
\begin{itemize}
  \item Zavolať majiteľovi - Keďže aplikácia zatiaľ nie je reálne používaná, nepovažovali sme za podstatné pridávať funkcionalitu kontaktovať administrátora aplikácie. Funkcionalitu môžeme v budúcnosti pridať.
\end{itemize}
\begin{figure}[H]
  \centering
  \includegraphics[width=.7\textwidth]{img/dotaznik/6.png}
\end{figure}
\indent Technické problémy:
\begin{itemize}
  \item Bug v príručke, PDF vidím cca v hornom 20\% okna - Chyba nastala pri interpretácií \acrshort{html} kódu prehliadačom \textit{Mozilla Firefox}. Chyba je už odstránená
\end{itemize}
\begin{figure}[H]
  \centering
  \includegraphics[width=.7\textwidth]{img/dotaznik/8.png}
\end{figure}
\begin{figure}[H]
  \centering
  \includegraphics[width=.7\textwidth]{img/dotaznik/9.png}
\end{figure}
\indent Iné podobné aplikácie:
\begin{itemize}
  \item strava - Aplikácia neslúži primárne na zobrazovanie trás. Áno, dajú sa v nej zobraziť trasy. Trasa sa daá zobraziť len jedna naraz, čo znamená že mapa bude vždy prehľadná.
  \item apple watch - Apple watch nie je aplikácia, možno mal testér na mysli Apple Health, ktoré neslúži na zobrazovanie vlastných trás. V Apple Health sa dá zobraziť trasa, no dá sa zobraziť len jedna naraz. Taktiež to nemôže byť ľubovoľná trasa, ale trasa nasnímaná počas tréningu. Apple Health neponúka map-match.
\end{itemize}
\begin{figure}[H]
  \centering
  \includegraphics[width=.7\textwidth]{img/dotaznik/11.png}
\end{figure}

Na základe odpovedí z dotazníka môžeme povedať, že aplikácia funguje správne. Nastal jeden technický problém. Používateľská príručka nebola správne zobrazená na stránke v prehliadači \textit{Mozilla Firefox}. Aplikáciu sme testovali na inom prehliadači (\textit{Google Chrome}). Vďaka testovaniu aplikácie a dotazníku sme na tento problém narazili a odstránili sme ho. Jeden z testérov navrhol, aby sme po prvom prihlásení hneď otvorili okno nahrávania trás, keďže je zbytočné zobrazovať mapu bez toho aby boli nahrané trasy, ktoré by sa na nej dali zobraziť. V budúcnosti môžeme toto správanie implementovať. 