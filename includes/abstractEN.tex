The goal of this work is to design and implement an algorithm that will visualize larger datasets of GPS trajectories clearly. This means that the user will be able to recognize basic movement patterns by sight from the trajectory. The work output will be displayed interactively in a web browser. We will create a web application in which the user authenticates and uploads his GPS trajectories. Once uploaded, these trajectories are processed by the algorithm and saved to server storage. The user will be able to view and compare these adjusted trajectories with the unmodified trajectories. Trajectories will be displayed in an interactive map that can be zoomed in or out. The work consists of seven chapters. The first chapter deals with issues of displaying GPS trajectories. The second chapter deals with the issue of web application development. In the third chapter, we will present the available solutions, their advantages and disadvantages. In the fourth chapter, we will present the technologies we used to implement the algorithm and implement the web application. In the fifth chapter, we define the functional and non-functional requirements of the application and present a diagram of the use cases of the application, along with sequence diagrams of some processes. We will also propose a way to store data on the server and how to edit trajectories. In the sixth chapter, we deal with the implementation of the web application and editing trajectories. In the last, seventh chapter, we evaluate the created application and the quality of route modification based on the answers to the questionnaire from the application testers.