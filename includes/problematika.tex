\section{Problematika vývoja webových aplikácii}
\subsection{Webové aplikácie a ich vývoj}
Webové aplikácie a ich vývoj je veľmi dôležitý v oblasti informačných technológií. Umožňuje vývojárom tvoriť a distribuovať softvér cez internet. Výhodou Webových aplikácií je, že poskytujú interaktívne a dynamické prostredie pre používateľov prostredníctvom internetových prehliadačov. Pri vývoji webových aplikácii je možné sa stretnúť s mnoho technológiami a programovacími jazykmi\cite{create_web_app}:
\begin{itemize}
    \item \textbf{HTML/CSS}  Patria medzi základné jazyky, pomocou ktorých sa tvorí obsah webových stránok.
    \item \textbf{Javascript} Skriptovací jazyk, ktorý zaručuje dynamickosť webovej stránky, používateľ môže mať pomocou tohto jazyka interkaciu so stránkou.
    \item \textbf{Backend jazyk} Jazyk, ktorý je použitý na serverovú časť webovej aplikácie, ako napríklad prihlásenie, prácu s databázou a inej logiky na pozadí, ktorú používateľ nevidí. Patria sem napríklad: Javascript, Python, PHP, Java a ďalšie.
    \item \textbf{Databázy} V mnoho aplikáciach treba dáta ukladať a tieto dáta sa ukladajú v databázach. Uloźené dáta je možné prehľadávať alebo inak s nimi manipulovať. Medzi rôzne databázy patria napríklad MySQL, PostgreSQL, MongoDB a ďalšie.
    \item \textbf{Knižnice a frameworky} Používajú sa pre zjednodušnie vývoja a zlepšenie efektivity. Patria sem napríklad Angular, React, Vue.js pre frontend alebo Django, Laravel, Express, Flask pre backend.
\end{itemize}
\subsection{Vývojový proces v 8 krokoch:}
\begin{enumerate}
    \item \textbf{Prieskum trhu}: V prvom rade treba zistiť, aký problém používateľ má. Treba naštudovať existujúce riešenia problému. Pri existujúcich riešeniach treba zvážiť výhody a nevýhody riešení. Vo vlastnej webovej aplikácii sa budeme snažiť nevýhody odstrániť.
    \item \textbf{Definovanie funkcionality}: Aplikácia musí mať určené, aké funkcionality bude ponúkať. Tieto funkcionality majú riešiť problém používateľa. Chceme dosiahnuť, aby naša aplikácia riešila všetky problémy používateľa a aby nemala zbytočnú funkcionalitu, ktorú používateľ nebude používať a bude nevyužitá. Touto zbytočnou funkcionalitou sa proces vývoja zbytočne predĺži. 
    \item  \textbf{\acrshort{mvp}}: Vytvorenie základnej verzie aplikácie, kde sa nedáva špeciálny dôraz na dizajn. Je to návrh ako by mala aplikácia fungovať s implementáciou hlavných funkcionalít.
    \item \textbf{Databáza}: Vytvorenie a pripojenie aplikácie na databázu, kde sa budú dáta bezpećne ukladať pre neskoršie načítanie a používanie v aplikácii.
    \item \textbf{Frontend}: V tomto kroku vytvoríme dizajn aplikácie. S týmto dizajnom budú používatelia pracovať, preto by mal byť prehľadný, pekný a zaujímavý.
    \item \textbf{Backend}: Vytvorenie serverovej časti. Patrí sem komunikácia s databázou, výpočty na pozadí, ktoré používateľ nevidí.
    \item \textbf{Testovanie}: Testovanie aplikácie, či funguje správne. Odskúšame funkcionalitu aplikácie a odhalíme prípadné chyby a zistíme, či sa aplikácia správa podľa očakávaní.
    \item \textbf{Nasadenie}: Nahranie aplikácie na server, aby ju mohol používať používateľ odkialkoľvek.
\end{enumerate}


\subsection{Výber technológií}

\noindent\textbf{Javascript}\cite{javascript}\\
Javascript je jedným z najpopulárnejších a najuniverzálnejších programovacích jazykov pre vytváranie webových aplikácií. Umožňuje dynamické a interaktívne funkcie, ktoré zlepšujú používateľskú skúsenosť s aplikáciou a funkčnosť webových stránok. Môže bežať na strane klienta aj na strane servera. Je to interpretovaný jazyk. Má bohatý ekosystém knižníc a frameworkov, ktoré poskytujú rôzne nástroje a abstrakcie pre vývoj webových aplikácii. Jedným z týchto frameworkov je aj Express, ktorý použijeme na vytvorenie serverovej časti. Express bližšie opíšeme v sekcii \ref{section:express}.

\noindent\textbf{Visual Studio Code}\\
Spomedzi viacerých vývojových prostredí a textových editorov, v ktorých je možné vytvárať webové aplikácie sme si vybrali \textit{Visual Studio Code}. \textit{Visual Studio Code} je textový editor, ktorý podporuje množstvo rozšírení a programovacích jazykov. Prináša mnoho výhod, ktoré sú opísané nižšie v sekcii \ref{section:vscode}.

\noindent\textbf{MySQL}\\
Aplikačné údaje, napríklad údaje o registrovaných používateľoch, informáciu o prihlásení a podobne ukladáme do aplikačných databáz. Tieto databázy umožňujú štrukturalizáciu dát do tabuliek, čo pomáha efektívnej správe informácií. \textit{MySQL} je najpopulárnejšia open-source databáza\cite{mysql}. Použijeme ju na uloženie informácií o používateľoch a na uloženie informácie pre správcu aplikácie v prípade zlého pripnutiu trasy k cestnej sieti. Viac o \textit{MySQL} je popísané v sekcii \ref{section:mysql}.

\noindent\textbf{Valhalla}\\

\noindent\textbf{Leaflet}\\