Cieľom tejto práce je navrhnúť a implementovať algoritmus, ktorý bude vizualizovať väčšie datasety GPS trajektórií prehľadne. To znamená, že používateľ bude schopný rozlíšiť základné pohybové vzory zrakom z trajektórií. Výstup práce bude zobrazený interaktívne vo webovom prehliadači. Vytvoríme webovú aplikáciu, v ktorej sa používateľ autentifikuje a nahrá svoje GPS trajektórie.  Po nahratí sa trajektórie spracujú algoritmom a uložia sa na serverové úložisko. Používateľ bude môcť upravené trajektórie zobraziť a porovnať s neupravenými. Trajektórie budú zobrazené v interaktívnej mape, ktorá sa bude dať priblížiť a oddialiť. Práca sa skladá zo siedmych kapitol. Prvá kapitola sa venuje problematike zobrazovania GPS trajektórií. Druhá kapitola sa venuje problematike vývoja webových aplikácií. V tretej kapitole si predstavíme dostupné riešenia, ich výhody a nevýhody. V štvrtej kapitole predstavíme technológie, ktoré sme použili pri implementovaní algoritmu a implementovaní webovej aplikácie. V piatej kapitole definujeme funkcionálne a nefunkcionálne požiadavky aplikácie a predstavíme diagram prípadov použitia aplikácie, spolu so sekvenčnymi diagramami niektorých procesov. Taktiež navrhneme spôsob akým budeme ukladať údaje na server a ako budeme upravovať trajektórie. V šiestej kapitole sa venujeme implementácií webovej aplikácie a úprave trajektórií. V poslednej, siedmej kapitole hodnotíme vytvorenú aplikáciu a kvalitu úpravy trás na základe odpovedí na dotazník od testérov aplikácie.